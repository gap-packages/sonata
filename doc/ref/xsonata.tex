%
\Chapter{Graphic ideal lattices (XGAP only)}
%

If you run SONATA under XGAP or Gap.app, it is possible to study ideal lattices of
nearrings graphically.

\>GraphicIdealLattice( <nr>, <string> )

The function `GraphicIdealLattice' computes one- and two-sided
nearring ideals and uses XGAP's graphic capabilities to draw the ideal
lattice of the nearring <nr>. The string <string> determines, which
ideals are shown:

If <string> contains the letter `l', left ideals are shown, if
<string> contains the letter `r', right ideals are shown, and if
<string> contains the letter `i', two-sided ideals are shown.  Any
combination of these letters is possible.

Right ideals of the nearring are represented by squares, left ideals
by diamonds and two-sided ideals by circles. It is possible, that
two-sided ideals are shown as right or left ideals, if the two-sided
ideal property has not yet been tested.

Left clicking on an ideal allows one to select the ideal and determine
more information about the ideal. Choosing `Ideal type' from the
`Ideals'-menu determines, whether the selected ideal is two-sided (in
which case its shape changes to a circle immediately). On a right
click on an ideal a window opens showing more information about the
ideal, such as its size and the isomorphism class of the factor
nearring by the ideal (if it is two-sided). The information is
computed and displayed as soon as you click on the corresponding entry
in the window. Finally clicking on `Export ideal to GAP' makes the
ideal the last output of XGAP's main window, where you may then go on
working with it.

It is also possible to select several ideals at the same time pressing
the Ctrl-key while left clicking on the ideals. When several ideals
are selected it is possible to compute their sums and intersection
from the corresponding `Ideals'-menu entries. The result of the
computation is then indicated by green color.

%%% Local Variables: 
%%% mode: latex
%%% TeX-master: t
%%% End: 
